\documentclass[a4paper,10pt]{article}
%\documentclass[a4paper,10pt]{scrartcl}

\usepackage[utf8x]{inputenc}

\title{Machine Learning CBC: Assignment Two}
\author{csl209}
\date{}

\usepackage{mathptmx}

\pdfinfo{%
  /Title    ()
  /Author   ()
  /Creator  ()
  /Producer ()
  /Subject  ()
  /Keywords ()
}

\begin{document}
\LARGE\noindent
\textsc{Machine Learning CBC \\ Assignment Two} \\[1cm]

\large\noindent
Xin Yan Kwek, Darcy Qiu, Charles Koh, and C.S. Rudolf Lai\\ 

\normalsize

\section*{Pruning}

The \texttt{pruning\_example} function executes three \texttt{MATLAB} tree functions in this sequence: \texttt{treefit}, \texttt{treetest}, and \texttt{treeprune}. First it uses \texttt{treefit} to create a decision tree for predicting responses (targets) as a function of predictors (examples).  The tree is created such that impure nodes will be split as long as there are observations. Using \texttt{treetest}, the expectation of misclassification costs over all terminal nodes is then computed using both the resubstitution method and the 10-fold cross-validation method. Along with the cost, the standard error of each cost value, the terminal node count for each subtree, and an \textit{estimated best level of pruning} is also computed. Using the two estimated best level, we then use \texttt{treeprune} to prune the tree, coming up with two new trees.
\end{document}
